%\section{Physical properties of the edge plasma}\label{sec:props}
The following scrape-off layer (SOL) parameters (including decay lengths) are for the
L-mode scrape off layer in MAST~\cite{Mi13Expe}.
For MAST,  with the standard notation,
$R=1.6$\,m, $B_T \approx 0.6$\,T, $I_\phi \approx 400$\,kA and
$a=0.6$\,m, values which imply that the poloidal field at the
plasma edge $B_p \approx 0.1$\,T.
The main result of the paper~\cite{Mi13Expe} is that the decay length of the power
deposition at midplane is $\lambda_q \approx 2$\,cm (range $1-3$\,cm)
and $P_{tot} \approx 350$\,kW.

Unless stated otherwise, the derived length, timescales and speeds are derived from
formulae and graphs in Wesson's book~\cite[Chap 10]{wesson}.
The derived quantities have been checked against SI formulae
in ref~\cite[Table 2.2]{miyamoto}, and also compared with those
listed in~\cite[Appendix]{Xu10Inte}. It is worth noting that although
the latter table describes the SOL of JET (and in addition its separatrix and pedestal),
JET values are typically within a factor of~$2$ of those for MAST.

\subsection{Typical discharge}

The edge values found experimentally are 
$T_e \approx 10$\,eV, $T_i \approx 20$\,eV, $n \approx 3 \times 10^{18}$\,m$^{-3}$.
These imply that the Coulomb logarithm $\Lambda \approx 12.5$, and
the flow speed~$U_d \approx 10^5$\,ms$^{-1}$ may be estimated using
$P_{tot}= 2 \pi R \lambda_q n (T_e+T_i) U_d$.
Sadly there appears to be no reliable determination of the neutral density~${\sf n}$.
(Note use of different font to distinguish neutral density from plasma density.)

\subsection{Length scales}

Debye length $\lambda_D \approx 10^{-5}$\,m. \\
Electron Larmor radius $\rho_{te} \approx 7 \times 10^{-5}$\,m. \\
$\rho_{ti} \approx 40 \rho_{te} \approx 4$\,mm. \\
Mean free path for electrons $\lambda_{emfp} \approx 1$\,m (parallel to field).

\subsection{Time scales}

%$\tau_e \approx 2 \times 10^{-8}$\,$s$.
Collision frequency (electrons with ions) $\nu_e \approx 3$\,MHz, $\tau_e \approx 3 \times 10^{-7}$\,s~\cite{NRLpf07}. \\
Plasma frequency $f_{pe} \approx 15$\,GHz, $\tau_{pe} \approx 7 \times 10^{-11}$\,s. \\
$f_{pi} \approx 0.4$\,GHz, $\tau_{pi} \approx 3 \times 10^{-9}$\,s. \\
Cyclotron frequency based on $B_p=0.1$\,T, $f_{ce} \approx 2.8$\,GHz, $\tau_{ce} \approx 4 \times 10^{-10}$\,s. \\
$f_{ci} \approx 1.4$\,MHz, $\tau_{ci}\approx  7\times 10^{-7}$\,s.

\subsection{Speeds}
Electron thermal $c_{se} \approx 1.2 \times 10^6$\,m$s^{-1}$. \\
$c_{si} \approx 4 \times 10^4$\,m$s^{-1}$. \\
Alfven speed using $B_T$ is $U_A \approx 10^7$\,m$s^{-1}$.

Collisionality parameter $\nu^{*}_c = \frac{q_e^4}{3 m_p^2\epsilon_0^2} L_0 n_0 /C_0^4$ \\
(note that $\frac{m_p^2\epsilon_0^2}{e^4}=\frac{1}{3}\;s^4 m^{-6}$.)\\
Taking $L_0 \approx 10$\,m, $n_0=n$.  Squared sound speed $C_0^2= T_i (|e|/m_e) (m_e/m_i)$, $C_0 \approx 3 \times 10^4$\,m$s^{-1}$, implies  \\
Collisionality parameter $\nu^{*}_c \approx 30$. \\
Peclet number $\approx 0.4 \nu^{*}_c \approx 10$, but turbulent coefficients$\approx1$\,m$^2$s$^{-1}$
will generally give a smaller value.

Resistive diffusion $\eta_d=15$\,$m^2 s^{-1} \propto T_e^{-3/2}$. \\
(Note that there is a notational clash with~$\eta$; fusion physics and astrophysics
differ by a factor~$\mu_0$, so that $\eta_d=\eta(\mbox{fusion})/\mu_0$.)

\subsection{Applicability of Fluid Models}\label{sec:applmhd}
A key requirement for fluid models is that collision times should be much less than
the timescale of interest, which as the preceding subsections show is true, except in the case of
$\tau_e$, the electron-ion collision time, and for the electrons more generally
for dynamics along the field-lines. The ion gyroradius is also uncomfortably large
compared to quantities of interest. Note that $\tau_e$ is the longest timescale
in the classical picture of approach to a single fluid picture of plasma, other
timescales, including the timescale for momentum to equilibrate, are shorter.

Single fluid MHD is widely used in astrophysics consistent with the eloquent advocacy
by Priest and Forbes~\cite[\S\,1.7]{priestforbes}. They point out that ideal MHD
is consistent with the drift ordering, despite confusion caused by the easy possibility
to misinterpret
Hazeltine and Meiss~\cite {hazeltinemeiss} on the subject. (The point is that
although MHD treats a faster timescale, it is valid on longer timescales, provided
relevant smaller/slower terms are retained.)
Moreover, SOL timescales involving filaments are fast, witnessed by the fact that
the ion gyro-frequency is used as normalisation for electrostatic models in
ref~\cite{Mi12Simu}, which from \Sec{props} is a not too dissimilar timescale~$10^{-7}\,s$
to the Alfven timescale based on the poloidal field~($1$\,cm/$10^6 \approx 10^{-8}$\,s).
Later, Freidberg~\cite{freidberg}
showed that, at least in directions perpendicular to~${\bf B}$, the dynamical MHD equation
applies to a more general `guiding centre' plasma. The situation may be summarised
by saying that complexity lies mostly in the transport (diffusive) terms
as these attempt to account for low collisionality, finite Larmor radius (FLR) etc.

Perhaps fortunately, the terms predicted by kinetic theory will usually be small
(except for the electrical conductivity) compared to the turbulent transport
expected on the basis of both observation and theory of the SOL plasma. The
simplest way to account for turbulence is to assume ad-hoc isotropic, uniform `eddy'
diffusivities in addition to the usual fluid advection terms.
Lastly, in a simple extension of MHD, large~$\tau_e$  is accounted for by allowing
the electrons and ions to have different temperatures, consistent with observation.
Effects due to the presence of a large neutral population in the SOL
could well be significant, see next \Sec{neuts}.
However neutrals are mainly expected to act as a sink of momentum and energy.

\subsection{Effect of Neutrals}\label{sec:neuts}
Formulae for a weakly ionised plasma are given in the Plasma Formulary\cite{NRLpf07}.
The collision cross-sections for electrons and ions respectively from
ref~\cite{Ha91hydr} are $\sigma_s^{e|0}= 10^{-19}$\,$m^2$ and
$\sigma_s^{i|0}= 4 \times 10^{-19}$\,$m^2$.
Hence, the collision frequencies for electrons and ions respectively are
\begin{equation}
\nu_{e{\sf n}}= 1.2 \times 10^{-13} {\sf n},\;\;\; \nu_{i{\sf n}}= 2 \times 10^{-14} {\sf n}
\end{equation}
where ${\sf n}$ is the neutral density.
(Note use of different font to distinguish neutral density from plasma density.)
If ${\sf n}=n$ is assumed, then
the corresponding SOL collision times are
\begin{equation}
\tau_{e{\sf n}}= 3 \times 10^{-6} s,\;\;\; \tau_{i{\sf n}}= 2 \times 10^{-5} s
\end{equation}
so that the number of collisions experienced by a typical SOL ion
before it hits a PFC is small.
Nonetheless, since $1/m_e \gg 1/m_i$, $D_e \gg D_i$ and the
diffusion coefficient for both electrons and ions is numerically large
\begin{equation}
D_A \approx (1+\frac{T_e}{T_i}) D_i \approx \frac{10^{23}}{{\sf n}}
\end{equation}

The parallel electrical diffusivities are different,
for electrons and ions respectively these are
\begin{equation}
\eta_{e {\sf n}\parallel} =4 \frac{{\sf n}}{n},\;\;\;
\eta_{i {\sf n}\parallel} =0.5 \frac{{\sf n}}{n}
\end{equation}
The implication from the formulae in ref~\cite{Le06emer} is that 
the value for $\eta_{e \parallel}$ combines additively with the usual
Spitzer value in a more
highly ionised plasma. Assuming ${\sf n} \approx n$, however the correction
is seen to be an increase of $4$ in $15$\,$m^2s^{-1}$, i.e. only about~$25$\,\%.

Arber~\cite{Le06emer,Ar07Emer} further points out that
according to the Formulary~\cite{NRLpf07},
in a weakly ionised plasma
the conductivity is greatly reduced (and the magnetic diffusivity
correspondingly enhanced) in directions normal to a strong magnetic field.
Typically for Braginskii theory, the
factor is $x_e^2$ for the perpendicular direction and $x_e$ for
the other direction, where
\begin{equation}
x_e=\frac{2\pi f_{ce}}{\nu_{e{\sf n}}} \approx \frac{8 \times 10^{23}}{{\sf n}}
\end{equation}
For ${\sf n}=n=3 \times 10^{18}$, these are huge increases. However it is worth
noting that if the electromagnetic potential representation is invoked, so that
\begin{equation}\label{eq:E}
{\bf E} = -\nabla \Phi + \frac{\partial {\bf A}}{\partial t}
\end{equation}
then, in the direction parallel to~${\bf B}$, neglecting the gradient of
electric potential~$\Phi$
\begin{equation}\label{eq:A}
\frac{\partial  A_{\parallel}}{\partial t}=\eta_{e {\sf n}\parallel} J_{\parallel}
\end{equation}
Thus the enhanced diffusivities need not signify if this equation is used for
magnetic field evolution, although applying a gauge condition on the potentials
may become difficult in complicated 3-D topologies.

