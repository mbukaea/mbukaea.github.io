%\section{Documentation Generally}\label{sec:OP_MGT}

\subsection{Documentation} \label{sec:doc}

Documentation refers to a particular version of the code. It should therefore be
dynamic, under version control, and tightly coupled to the source code itself. 

\begin{itemize}
\item All new code features should be documented, and this should be checked as
part of the peer review process.
\item Within the code, comments should use a convention, such as that
accepted by \F{Doxygen},
to document the intent of functions, and any assumptions on their
environment, input or outputs.
\item Alongside the code \T{README} files explaining the file/directory layout 
typically use the Markdown format due to its simplicity,
standardising on the variant defined by \F{Pandoc} as described in \cite{y2d34}.
\item The more formal documentation should be in a format which can
include elements such as equations, code blocks, graphs and figures.
It should also be easily convertible to other formats, and in
particular online documentation. \LaTeX \ as used to produce the current document
can be easily converted to .html as explained in ref~\cite{y2d34} provided
the restrictions (as to accepted packages) noted in the reference are observed.
\end{itemize}

