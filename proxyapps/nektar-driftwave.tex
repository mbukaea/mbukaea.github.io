\newsection{Nektar-Driftwave}{sec:NektarDriftware}
Nektar-Driftwave solves the 2D Hasegawa-Wakatani equations describing drift-wave turbulence.  The equations are, for
vorticity $\zeta$, number density perturbation $n$, and electrostatic potential $\phi$,

\begin{eqnarray}
\frac{\partial \zeta}{\partial t}+[\phi, \zeta] &=& \alpha (\phi-n)\\
\frac{\partial n}{\partial t} +[\phi,n] &=& \alpha (\phi-n)-\kappa \frac{\partial \phi}{\partial y},
\end{eqnarray}

where $[a,b] \equiv \frac{\partial a}{\partial x} \frac{\partial b}{\partial y} - \frac{\partial a}{\partial y}
 \frac{\partial b}{\partial x}$.

In the above, $\alpha$ is the adiabaticity operator (taken to be constant) and $\kappa$ is the background density 
gradient scale length.  The electrostatic potential is related to the vorticity by Poisson's equation $\nabla^2 \phi 
= \zeta$.

The main system is implemented as an advection problem using a discontinuous Galerkin formulation which provides numerical
stabilization meaning that the usual hyperviscosity term is not required.  The Poisson solve is implemented in a 
continuous Galerkin formulation. 

The example provided tracks the nonlinear evolution of an initial Gaussian spatial density perturbation to a
 turbulent quasi-steady state.  See the internal report \cite{y3re222} for a presentation of the output and a 
comparison with published results.

Nektar-Driftwave is publicly available at \url{https://github.com/ExCALIBUR-NEPTUNE/nektar-driftwave}.  
