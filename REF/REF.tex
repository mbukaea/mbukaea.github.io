% Sommerville  - Glossary
% Smith  - Requirements  - Reference Material
% Smith  - Requirements  - Specific System Description  - Problem Description  - Terminology and Definitions
% Hewitt  - Infrastructure Design  - Standards and guidelines and conventions
% Hewitt  -  Acronyms and Symbols
\newsectionnobreak{Conventions for Report Writing}{sec:conv}
The project has strict standards in respect of conventions even in reports
that contain no code, ultimately to support the `write once, use many times' concept,
so that text may be cut, edited and pasted into source code, or indeed used safely to define
variable names, constraints on their values and physical dimensions.
Thus, since many compilers support only ASCII characters, only ASCII is allowed
in reports. Similarly, very long lines (typically arising from MS Windows wordprocessing)
should be split at the ends of sentences, and preferably so that maximum line length
is $120$~characters. A number of tools are available in \T{tex/importools}.
Contributors may use their style of choice for both reports (and code) provided
they follow systems that allow for automatic conversion to the conventions
expected in the \nep \ repositories, and are prepared to help construct gitlab runners and
github hooks to this end. However, it will be generally found easier to use \LaTeX \ and
bibtex as indicated below.

For shorter documents it is acceptable to use Markdown in the `dialect' defined by
\F{PANDOC}, see Annex~A of~\cite{y3re314}. Resulting .md files frequently convert
straightforwardly to \LaTeX \ format, using the md2tex.bash script from \T{tex/importools}.


\subsection{General textual issues}\label{sec:REF_text}
It is helpful to use \LaTeX \ newcommand for certain keywords where they have a special
format or because the terminology is not fully established. Although some do not, most
people do find variations in use of spelling, punctuation, capitalisation etc.\ to be irritating,
and so the following conventions will be enforced where possible: 
\begin{itemize}
\item use of \LaTeX \ newcommand for , viz. \verb-\-papp \ and  \verb-\-exc \ instead of explicit
proxyapp or mini-app and \exc \  respectively
\item punctuation, viz.\ eg.\verb-\- rather than e.g.
\item hyphenation, generally to be avoided because \LaTeX \ may use it to break lines, viz.\ finite element rather than finite-element (or simply FE), opensource rather than open source, major exception is abbreviation of dimensions (below).
\item capitalisation, eg.\ Exascale rather than exascale
\item abbreviation of dimensions, 2-D and 3-D preferred to 2D and 3D or 2d and 3d
\item spelling, UK English preferred, and  '-ise' etc.\ preferred to '-ize'
\item no spaces between authors' initials in bibtex files (see also \Sec{REF_citations})
\item consistent usage of acronyms, employing the table of \Sec{acro}
\item consistent usage of mathematical symbols, employing the table of \Sec{symbol}
\end{itemize}

\subsection{Citations}\label{sec:REF_citations}
Regarding citations, those for unpublished reports MUST include a link to a
website, such as arXiv, and it would be helpful if links to other open-access
material were also given.
Use of the following conventions for constructing the keys of citations should help avoid clashes:
\begin{enumerate}
\item For published papers, use where possible an $8$-character alphanumeric for published papers,
consisting of the first two letters of the first author's name, the last two digits
of the year in the Gregorian calendar, and the first $4$~letters of the first \emph{significant}
word (ie.\ not `The' or~`On') of the title, preserving capitalisations.
Thus the paper~\cite{Ba13What} by Bangerth and Heister, published in 2013 and
entitled `What makes computational open source software libraries successful?',
has key~`Ba13What'.
\item Books and theses should be keyed with the full second-name(s) of the author(s) strung together
without capitalisation, up to a limit of approx.~$30$ characters, using `etal'
to indicate any omitted author-names. As an example, the book by Rouson, Xia
and~Xu~\cite{rousonxiaxu} has key~`rousonxiaxu'.
\item If there are duplicates in different files, then preface
each key with the name of the .bib file it is in, for other duplicates within a file,
add `2', `3' etc.\ to the end of the key.
\end{enumerate}

\subsection{Software Compatibility}\label{sec:REF_compat}
% to TS % Latest C++, C++20 if possible including modules. (like Fortran-95 only Import instead of use), and
% to TS %             Concepts - generalised types of single variable.
% to TS % Follow Stroustrop~\cite{stroustroptour}.
% to TS % Use of clang\_tidy.

When discussing software in the text, the following conventions in \LaTeX\  are proposed:
\begin{itemize}
\item \F{Small capitals} denote a package name, use \{\verb-\-textsc or abbreviation \verb-\-F\{
\item \I{Italics} denote a program name, use \{\verb-\-textit or abbreviation \verb-\-I\{
\item \T{Fixed width font} denotes any code name or fragment which is not otherwise obviously source code,
use \{\verb-\-texttt or abbreviation \verb-\-T\{
\end{itemize}
There is no need for special fonts if the object is identified by a
suitable suffix, thus ``\_m" for a module containing executable code,
``\_h" or ``.h" for an object description or namespace code, ``.cpp" for name of file containing
C++ source, ``.exe" for an executable, etc. Similarly file suffices that imply a particular
format or software for their interpretation, may simply be written with a leading stop,
eg.\ ``.html" and ``.exe".

In order to ensure smooth transliteration from mathematical symbols to the names
of the software variables, \Tab{twoclatex} lists the recommended two-character
equivalents for \LaTeX \ symbols used in the definition of mathematical symbols.
\begin{table}[tbph]
\begin{center}
\caption{\textbf{\textsf{TWO CHARACTER EQUIVALENTS}} of \LaTeX \ symbols and commands.  \label{tab:twoclatex} }
\begin{tabular}{||p{1.5cm}|p{4cm}||p{1.5cm}|p{4cm}||p{1.5cm}|p{4cm}||}
\hline
aa & $A$ & al & $\alpha$ & ar & $\rightarrow$ \\
as & $*$ & bb & $B$ & be & $\beta$ \\
bl & $[$ & br & $]$ & cc & $C$ \\
ch & $\chi$ & ci & \verb-^- & dd & $D$ \\
de & $\delta$ & dl & $\Delta$ & dq & " \\
ds & \verb-\-ddot & dv & $/$ & ee & $E$ \\
el & $\ell$ &  &  & &  \\
ep & $\epsilon$ & et & $\eta$ & ff & $F$ \\
ga & $\gamma$ & gg & $G$ & gm & $\Gamma$ \\
gt & $>$ & hh & $H$ & ii & $I$ \\
in & $\infty$ & & & & \\
it & $\iota$ & jj & $J$ & ka & $\kappa$ \\
kk & $K$ & la & $\lambda$ & ll & $L$ \\
lm & $\Lambda$ & lt & $<$ & me & $\omega$ \\
mg & $\Omega$ & ml & $\times$ & mm & $M$ \\
mn & $-$ & mu & $\mu$ & n1 & $n_1$ etc. \\
nn & $N$ & nu & $\nu$ & o2 & \verb-\-boldmath \\
o3 & \verb-\-mathcal & o4 & \verb-\-mathsf & o5 & \verb-\-mathtt \\
o6 & \verb-\-mathbb & o7 & \verb-\-mathfrak & \\
oe &  suffix  & ol &  preceding superfix  & on &  above  \\
or &  superfix  & os &  underneath  & ow &  prefix  \\
pa & $\|$ & pe & $\perp$ & pf & $\Phi$ \\
ph & $\phi$ & pi & $\pi$ & pj & $\Psi$ \\
pl & \{ & pp & $P$ & pr & \} \\
ps & $\psi$ & pt & $\partial$ & pu & $+$ \\
py & $\Pi$ & qq & $Q$ & rh & $\rho$ \\
rr & $R$ & sg & $\Sigma$ & si & $\sigma$ \\
sq & $'$ & ss & $S$ & st & $.$ \\
ta & $\tau$ & te & $\Theta$ & th & $\theta$ \\
ti & $~$ & tt & $T$ & un & \_ \\
up & $\upsilon$ && && \\
us & $\Upsilon$ & uu & $U$ & vb & $|$ \\
ve & $\varepsilon$ & vf & $\varphi$ & vp & $\varpi$ \\
vr & $\varrho$ & vs & $\varsigma$ & vt & $\vartheta$ \\
vv & $V$ & ww & $W$ & xx & $X$ \\
yy & $Y$ & ze & $\zeta$ & zz & $Z$ \\
\hline
\end{tabular}
See \Sec{two-character-variables} for a guide explaining the reasons for the above choices.
\end{center}
\end{table}



\clearpage
\newsection{Acronyms}{sec:acro}
\begin{longtable}{|p{4.0cm}|p{12.0cm}|}
\caption{\textbf{\textsf{TABLE OF ACRONYMS}} Nearly all the acronyms refer to technical
terms. A debt is acknowledged to the book by Brunton and Kutz~\cite{bruntonkutz}.
} \\
\hline
ACM & Association for Computing Machinery\\
ADC & Analogue to digital converter\\
ADM  & Alternating directions method \\
AIC  & Akaike information criterion \\
ALM  & Augmented Lagrange multiplier \\
AMR & Adaptive mesh refinement\\
AMReX & Software framework for block-structured AMR\\
ANL & Argonne National Laboratory \\
ANN & Artificial Neural Network \\
ANOVA & Analysis of Variance \\
API & Application Programming Interface \\
ARMA  & Autoregressive moving average \\
ARMAX  & Autoregressive moving average with exogenous input \\
ASQ & Adaptive sparse quadrature \\
ATS & Advanced Terrestrial Simulator, previously Arctic Terrestrial Simulator \\
BC & Boundary Condition\\
BEIS &  (UK government) Department for Business, Energy and Industrial Strategy \\
BG/L & IBM Blue Gene / L supercomputer platform\\
BIC  & Bayesian information criterion \\
BOUT++ & Tokamak edge plasma modelling framework \url{https://boutproject.github.io/} \\
BPOD  & Balanced proper orthogonal decomposition \\
BSD  & Opensource software licence \\
%CabanaMD & \\
CAD & Computer-Aided Design, geometry including NURBS, usually ``CAD database" implied \\
CCA  & Canonical correlation analysis \\
CCFE & Culham Centre for Fusion Energy \\
CEA  & The French Alternative Energies and Atomic Energy Commission \\
CESM & Community Earth System Model\\
CFD  & Computational fluid dynamics \\
CI & Continuous integration\\
CLI & Command Line Interface \\
CNN  & Convolutional neural network \\
COGENT & LLNL continuum plasma simulation code\\
COMPAT & Computing patterns for multiscale HPC (project)\\
CoSaMP  & Compressive sampling matching pursuit \\
COSMO & Framework for regional weather prediction in Europe \\
COSSAN & UQ and risk analysis package (Uni. Liverpool)\\
CPP & C plus plus programming language\\
CPU & Central Processing Unit \\
CRUD & Create, Read, Update, Delete \\
CS & Compressed sensing \\
CSE & Computational science and engineering\\
CSG & Constructive Solid Geometry \\
CSMP & Computer science, mathematics, and physics\\
CTO & Chief Technology Officer \\
CUDA & Compute Unified Device Architecture \\
CWIPI & Coupling with interpolation parallel interface (coupling library)\\
CWT  & Continuous wavelet transform \\
DA & Data Assimilation\\
DAG & Direct Acyclic Graph \\
DAKOTA & UQ and optimization package (Sandia)\\
DCT  & Discrete cosine transform \\
DDA & Digital Differential Analyser\\
DDD & Document-Driven Design \\
DE & Differential equation \\
DEIM & Discrete Empirical Interpolation Method \\
DFT & Discrete Fourier Transform \\
DiMDc  & Dynamic mode decomposition with control \\
DL  & Deep learning \\
DMD  & Dynamic mode decomposition \\
DMDc  & Dynamic mode decomposition with control \\
DNS  & Direct numerical simulation \\
DOE & Department of Energy \\
DOI & Digital Object Identifier \\
DPC++ & Data Parallel C++, Intel compiler for C++ with SYCL extension \\
DRAM & Delayed Rejection Adaptive Metropolis \\
DSL & Domain-Specific Language \\
DWT  & Discrete wavelet transform \\
ECOG  & Electrocorticography \\
ECP & Exascale Computing Project \\
ECP-copa & Co-design centre for particle applications (part of ECP)\\
eDMD  & Extended DMD \\
EIM  & Empirical interpolation method \\
EIRENE & name of neutral package  \\
EM  & Expectation maximization \\
EOF  & Empirical orthogonal functions \\
ERA  & Eigensystem realization algorithm \\
ESC  & Extremum-seeking control \\
ESI & name of software company \url{https://www.esi-group.com/}  \\
ESMF & Earth System Modeling Framework \\
E-TASC & EUROfusion Theory and Advanced Simulation Coordination \\
ETS & European Transport Simulator\\
EU & European Union \\
FCI & Flux-Coordinate Independent (method) \\
FELTOR & name of edge code \\
FEM & Finite Element Method \\
FEniCS & name of PDE software project \url{https://fenicsproject.org} \\
FFT & Fast Fourier Transform \\
FFTW & Fastest Fourier Transform in the West (library) \\
FLASH & name of Multiscale physics code \\
GA & General Atomics \\
GBS & Global Braginskii Solver (software)\\
GCR & Generalied Collisional Radiative (framework) \\
GDB & Global Drift-Ballooning \\
GDB & GNU debugger \\
GDPR & General Data Protection Regulation \\
GENE & name of gyrokinetic code \\
GMM  & Gaussian mixture model \\
GMRES & Generalized Minimal Residual method \\
GNU & GNU's Not Unix! \\
GP & Gaussian Process \\
gPC & Generalised polynomial chaos (Xiu and Karniadakis \url{https://doi.org/10.1016/S0021-9991(03)00092-5} \\
GPU & Graphics Processing Unit \\
GRILLIX & name of 3D turbulence code based on the flux-coordinate independent approach \\
GSA & Global sensitivity analysis \\
GUI & Graphical User Interface \\
HAGIS & HAmiltonian GuIding centre System\\
HAVOK  & Hankel alternative view of Koopman \\
HDF5 & Hierarchical Data Format (version 5) \\
HDS & Hierarchical Data Structure \\
HLA & High Level Architecture\\
HPC & High Performance Computing \\
HTC & High Throughput Computing \\
IBM & International Business Machines Corp., but really known as IBM \\
IC & Initial Condition \\
ICA  & Independent component analysis \\
ICON & ICOsahedral Nonhydrostatic, the global numerical weather prediction model of the German weather service \\
IEEE & Institute of Electrical and Electronics Engineers \\
IETF & Internet Engineering Taskforce \\
IMAS & Integrated Modelling \& Analysis Suite, promoted by ITER \\
IMEX & Implicit-Explicit Methods \\
IO & Input/Output \\
ITER & name of International Thermonuclear Experimental Reactor \\
ITG & Ion Temperature Gradient \\
ITM & Ion Tearing Mode \\
ITPA & International Tokamak Physics Activity (ITER research programme)\\
JET & Joint European Torus \\
JIT & Just In Time \\
JL  & JohnsonLindensfrauss \\
JOREK & name of nonlinear MHD code\\
JSON & JavaScript Object Notation \\
KL  & Kullback Leibler \\
KLT  & Karhunen-Loeve transform \\
LAD  & Least absolute deviations \\
LAMMPS & Large-scale Atomic/Molecular Massively Parallel Simulator\\
LANL & Los Alamos National Laboratory \\
LASSO & Least Absolute Shrinkage and Selection Operator \\
LCFS & Last Closed Flux Surface \\
LDA  & Linear discriminant analysis \\
LGPL & GNU Lesser General Public License \\
LHSamp & Latin Hypercube Sampling\\
LLNL & Lawrence Livermore National Laboratory \\
LOO & Leave One Out  \\
LQE  & Linear quadratic estimator \\
LQG  & Linear quadratic Gaussian controller \\
LQR  & Linear quadratic regulator \\
LTI  & Linear time invariant system \\
MAP & Maximium A Posteriori \\
MBSE & Model-based systems engineering \\
MC & Monte-Carlo (methods) \\
MCMC & Markov chain Monte-Carlo \\
MCT &  Model Coupling Toolkit \\ % MCT is mentioned here and in a comment?
MD & Molecular Dynamics \\
MECE & Mutually exclusive and collectively exhaustive \\
MF & Multi-fidelity, Matrix-free \\
MFMC & Multi-fidelity Monte-Carlo \\
MHD & Magnetohydrodynamics \\
MIMC & Multi-Index Monte-Carlo \\
MIMO  & Multiple input, multiple output \\
MIS & Module Interface Specification \\
MIT & Massachusetts Institute of Technology \\
MIT licence& Opensource software licence~\cite{MITlicense} \\
ML & Machine Learning \\
MLC  & Machine learning control \\
MLMC & Multi-Level Monte-Carlo \\
MLMF & Multi-Level Multi-Fidelity\\
MMF & Multiscale Modeling Framework\\
MMS & Method of Manufactured Solutions \\
MOOSE & Multiphysics Object Oriented Simulation Environment\\
MOR & Model Order Reduction\\
%Most  & Common Acronyms \\
MPE  & Missing point estimation \\
MPI & Message Passing Interface \\
mrDMD  & Multi-resolution dynamic mode decomposition \\
MSSC & Materials Science and Scientific Computing\\
MUMPS & MUltifrontal Massively Parallel Sparse direct Solver\\
MUSCLE~3 & Multiscale Coupling Library and Environment version 3\\
NAG & Numerical Algorithms Group \\
NARMAX  & Nonlinear autoregressive model with exogenous inputs \\
NEMO & Nucleus for European Modelling of the Ocean\\
NEPTUNE & Neutrals and Plasma Turbulence Numerics for the Exascale \\
NetCDF  &  Network Common Data Form \\
NLS  & Nonlinear Schroedinger equation \\
NROY & Not ruled out yet \\
NUCODE & Software: SMARDDA/NUCODE for Neutral Beam Duct Calculations\\
NURBS & NonUniform Rational B-Spline \\
OASIS &  Ocean Atmosphere Sea Ice Soil\\
OASIS4 &  Ocean Atmosphere Sea Ice Soil version 4\\
ODE & Ordinary Differential Equation \\
OKID  & Observer Kalman filter identification \\
OLYMPUS & OLYMPUS Programming System\\
OMFIT & One Modeling Framework for Integrated Tasks\\
OneAPI & A Unified, Standards-Based Programming Model, \url{https://software.intel.com/en-us/oneapi}\\
OP2 & API with associated libraries and preprocessors for performance-portable parallel computations on unstructured meshes \url{https://github.com/OP-DSL/OP2-Common}\\
OpenMP & Open Multi-Processing \\
OU & Oxford University \\
OUU & Optimisation under uncertainty \\
PASTIX & Parallel Sparse matriX package\\
PBH  & PopovBelevitchHautus test \\
PC & Polynomial chaos \\
PCA  & Principal components analysis \\
PCE & Polynomial chaos expansion \\
PCP  & Principal component pursuit \\
PDE & Partial Differential Equation \\
PDE-FIND  & Partial differential equation functional identification of nonlinear dynamics \\
PDF  & Probability distribution function \\
PETSc & Portable Extensible Toolkit for Scientific Computation \url{https://www.mcs.anl.gov/petsc/}\\
PFC & Plasma Facing Component\\
PGD & Proper Generalised Decomposition \\
PIC & Particle-In-Cell\\
PICPIF & Particle-In-Cell-Particle-In-Fourier\\
PID  & Proportional-integral-derivative control \\
PIV  & Particle image velocimetry \\
POD & Proper Orthogonal Decomposition \\
POOMA & Parallel Object-Oriented Methods and Applications \\
PP20 & SIAM Conference on Parallel Processing for Scientific Computing 2020\\
PPMD & Performance-Portable Framework For Atomistic Simulations \\
PR & git Pull Request \\
PSyclone & PSyclone is a code generation system that generates appropriate code for the PSyKAl code structure developed in the GungHo project.  \url{https://github.com/stfc/PSyclone}\\
PyOP2 & Framework for performance-portable parallel computations on unstructured meshes \url{http://op2.github.com/PyOP2}\\
QA & Quality Assurance\\
QCG  & Quality in Cloud and Grid, see QCG Pilot Job\\
QMC & Quasi-Monte-Carlo \\
QoI & Quantity of interest \\
QoS & Quality of Service \\
RAID & Risks, Assumptions, Issues, Dependencies \\
RAJA & RAJA Performance Portability Layer (C++) \url{https://github.com/LLNL/RAJA} \\
REST & Representational State Transfer (Resources as simple CRUD objects) \\
RIP  & Restricted isometry property \\
RKF23 & Runge-Kutta-Fehlberg (\emph{aka} Embedded Runge-Kutta), $23$ denotes orders of scheme\\
RKHS  & Reproducing kernel Hilbert space \\
RMS & Root-mean-square \\
RNG & Random Number Generator\\
RNN  & Recurrent neural network \\
RO & Responsible Officer  \\
ROM & Reduced-Order Model \\
RPCA  & Robust principal components analysis \\
rSVD  & Randomized SVD \\
SAMRAI & Structured Adaptive Mesh Refinement Application Infrastructure\\
SD1D & name of 1-D edge code \\
SDLC & Software Development Life Cycle \\
SGD  & Stochastic gradient descent \\
SIAM & Society for Industrial and Applied Mathematics \\
SINDy  & Sparse identification of nonlinear dynamics \\
SISO  & Single input, single output \\
SLA & Service-level Agreement \\
SLE & Software Language Extensions \\
SLE & System Level Engineering \\
SLEPc & name of Scalable Library for Eigenvalue Problem Computations \\
SLSQT & Sequential Least-Squares' Thresholding \\
SMARDDA & name of Ray-tracing algorithm, hybrid of SMART and DDA \\
SMART & name of Ray-tracing algorithm based on use of octree\\
SMITER & SMARDDA modules with ITER interface \\
SNOWPAC & Stochastic Nonlinear Optimisation with Path-Augmented Constraints (software package) \\
SOL & Scrape-Off Layer \\
SOLEDGE & name of edge modelling code \\
SOLPS & name of edge modelling code combines B2 and EIRENE\\
SPH & Smoothed Particle Hydrodynamics \\
SRC  & Sparse representation for classification \\
SRO & Senior Responsible Owner role in UK  government project delivery \\
SRS & Software Requirements Specification \\
SSA  & Singular spectrum analysis \\
SSD & Scientific Software Development \\
StarPU & Runtime system supporting heterogeneous multicore architectures \url{http://starpu.gforge.inria.fr/doc/html/} \\
STARWALL & name of vacuum field code \\
STFT  & Short time Fourier transform \\
STIX & Scientific And Technical Information eXchange \\
STLS  & Sequential thresholded least-squares \\
STORM & Scrape-off layer Transport ORiented Module \\
STRUMPACK & STRUctured Matrix PACKage \\
SUNDIALS & name of ODE package \\
SVD & Singular Value Decomposition \\
SVM & Support Vector Machine \\
SYCL & C++-single-source heterogeneous programming for acceleration offload, \url{https://www.khronos.org/sycl/} \\
SysML & Systems Modeling Language  \\
TAE & Toroidal Alfven Eigenmode \\
TDD & Test Driven Development \\
TICA  & Time-lagged independent component analysis \\
TM & TradeMark\\
TOKAM & name of set of edge modelling codes\\
TOKAM3X & name of Edge modelling software \\
TOMS & Transactions on Mathematical Software \\
TORPEX & TORoidal Plasma Experiment \\
Trilinos & Object-oriented software framework for the solution of large-scale, complex multi-physics engineering and scientific problems \url{https://trilinos.github.io/}\\
TRIMEG & TRIangular MEsh based Gyrokinetic code\\
TSVV & Theory, Simulation, Validation and Verification, tasks of the E-TASC programme of Eurofusion \\
TUM & Technical University Munich\\
UK & United Kingdom \\
UKAEA & United Kingdom Atomic Energy Authority \\
UKRI & United Kingdom Research and Innovation, a non-departmental public body encompassing the research councils and Innovate UK \\
UML & Unified Modelling Language\\
UQ & Uncertainty quantification \\
US & United States \\
USA & United States of America\\
UTF-8 & Unicode Transformation Format (Unicode denotes Universal Coded Character Set) \\
UUID & Universally Unique IDentifier is a 128-bit label used for information in computer systems\\
VAC  & Variational approach of conformation dynamics \\
VDE & Vertical Dispacement Event\\
VECMA & Verified Exascale Computing for Multiscale Applications \\
VECMAtk & VECMA toolkit \\
VORPAL & name of Electromagnetic Particle-in-Cell code\\
VSVO & variable stepsize, variable order solver of differential equation \\
VVUQ & Verification, Validation and Uncertainty Quantification \\
XGC1 & name of Particle-based gyrokinetic code\\
XML &  eXtensible Markup Language\\
XMSF & eXtensible Modeling and Simulation Framework \\ 
XPN & \exc \  Project \nep \ \\ 
\hline
\end{longtable}


\clearpage
\newsection{Symbols}{sec:symbol}
\input{REF/symb}
