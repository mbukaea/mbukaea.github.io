% to TS % Latest C++, C++20 if possible including modules. (like Fortran-95 only Import instead of use), and
% to TS %             Concepts - generalised types of single variable.
% to TS % Follow Stroustrop~\cite{stroustroptour}.
% to TS % Use of clang\_tidy.

When discussing software in the text, the following conventions in \LaTeX\  are proposed:
\begin{itemize}
\item \F{Small capitals} denote a package name, use \{\verb-\-textsc or abbreviation \verb-\-F\{
\item \I{Italics} denote a program name, use \{\verb-\-textit or abbreviation \verb-\-I\{
\item \T{Fixed width font} denotes any code name or fragment which is not otherwise obviously source code,
use \{\verb-\-texttt or abbreviation \verb-\-T\{
\end{itemize}
There is no need for special fonts if the object is identified by a
suitable suffix, thus ``\_m" for a module containing executable code,
``\_h" or ``.h" for an object description or namespace code, ``.cpp" for name of file containing
C++ source, ``.exe" for an executable, etc. Similarly file suffices that imply a particular
format or software for their interpretation, may simply be written with a leading stop,
eg.\ ``.html" and ``.exe".

In order to ensure smooth transliteration from mathematical symbols to the names
of the software variables, \Tab{twoclatex} lists the recommended two-character
equivalents for \LaTeX \ symbols used in the definition of mathematical symbols.
\begin{table}[tbph]
\begin{center}
\caption{\textbf{\textsf{TWO CHARACTER EQUIVALENTS}} of \LaTeX \ symbols and commands.  \label{tab:twoclatex} }
\begin{tabular}{||p{1.5cm}|p{4cm}||p{1.5cm}|p{4cm}||p{1.5cm}|p{4cm}||}
\hline
aa & $A$ & al & $\alpha$ & ar & $\rightarrow$ \\
as & $*$ & bb & $B$ & be & $\beta$ \\
bl & $[$ & br & $]$ & cc & $C$ \\
ch & $\chi$ & ci & \verb-^- & dd & $D$ \\
de & $\delta$ & dl & $\Delta$ & dq & " \\
ds & \verb-\-ddot & dv & $/$ & ee & $E$ \\
el & $\ell$ &  &  & &  \\
ep & $\epsilon$ & et & $\eta$ & ff & $F$ \\
ga & $\gamma$ & gg & $G$ & gm & $\Gamma$ \\
gt & $>$ & hh & $H$ & ii & $I$ \\
in & $\infty$ & & & & \\
it & $\iota$ & jj & $J$ & ka & $\kappa$ \\
kk & $K$ & la & $\lambda$ & ll & $L$ \\
lm & $\Lambda$ & lt & $<$ & me & $\omega$ \\
mg & $\Omega$ & ml & $\times$ & mm & $M$ \\
mn & $-$ & mu & $\mu$ & n1 & $n_1$ etc. \\
nn & $N$ & nu & $\nu$ & o2 & \verb-\-boldmath \\
o3 & \verb-\-mathcal & o4 & \verb-\-mathsf & o5 & \verb-\-mathtt \\
o6 & \verb-\-mathbb & o7 & \verb-\-mathfrak & \\
oe &  suffix  & ol &  preceding superfix  & on &  above  \\
or &  superfix  & os &  underneath  & ow &  prefix  \\
pa & $\|$ & pe & $\perp$ & pf & $\Phi$ \\
ph & $\phi$ & pi & $\pi$ & pj & $\Psi$ \\
pl & \{ & pp & $P$ & pr & \} \\
ps & $\psi$ & pt & $\partial$ & pu & $+$ \\
py & $\Pi$ & qq & $Q$ & rh & $\rho$ \\
rr & $R$ & sg & $\Sigma$ & si & $\sigma$ \\
sq & $'$ & ss & $S$ & st & $.$ \\
ta & $\tau$ & te & $\Theta$ & th & $\theta$ \\
ti & $~$ & tt & $T$ & un & \_ \\
up & $\upsilon$ && && \\
us & $\Upsilon$ & uu & $U$ & vb & $|$ \\
ve & $\varepsilon$ & vf & $\varphi$ & vp & $\varpi$ \\
vr & $\varrho$ & vs & $\varsigma$ & vt & $\vartheta$ \\
vv & $V$ & ww & $W$ & xx & $X$ \\
yy & $Y$ & ze & $\zeta$ & zz & $Z$ \\
\hline
\end{tabular}
See \Sec{two-character-variables} for a guide explaining the reasons for the above choices.
\end{center}
\end{table}

