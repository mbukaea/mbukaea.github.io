%DESC %Audience for D3 are the developers, testers and support staff, plus 'occasional' manager.
%DESC %It is at a technical and precise level.
% Sommerville  - Preface
% Hewitt  - Program Name
\newsectionnobreak{Executive summary}{sec:exc-sum}

The program addresses key plasma modelling issues for reactor design.
This new software should become essential for designing optimal 
power-handling strategies at the tokamak first wall, both directly (via the 
simpler \papp s) and indirectly by improving detailed physical understanding of the 
often turbulent,  plasma-wall interaction.

This website is set out as described in~\cite{y2d34} 
to cover all aspects of the overall \nep \ package, as they emerge,
beginning with the initial concept, advancing to detailed design
of classes (objects) and  interfaces, and ultimately producing documentation
to ensure the software remains usable, maintainable and relevant for at least 30~years.


\subsection{Further information}
High-level project issues are covered by the project science plan~\cite{sciplan};
the \nep \ project is \exc \ funded~\cite{exch+eswebsite}. 
Distribution and use of the software is covered by the extremely
permissive MIT licence~\cite{MITlicense}, regarded as equivalent to the BSD3 licence.
Collaboration is encouraged, and there are many additional benefits of community membership
as set out below. Those interested in joining the community should
 email {\tt neptune@ukaea.uk} to establish a dialogue.


\subsection{Benefits of community membership}
The principal benefit is access to what should ultimately become a powerful 
and comprehensive software package for modelling tokamak edge plasma using
finite element and particle methods.
In addition members will also gain
\begin{itemize}
\item Access to reports as the community produces them, in the access-controlled github site~\cite{xpndocswebsite},
subdirectories {\tt reports} and {\tt reports/ukaea\_reports}.
These directories already include educational material on
\begin{itemize}
\item Finite elements
\item Surveys of current software
\item Surveys of current HPC machines and performance
\item Uncertainty Quantification
\item Aspects of software engineering, such as design patterns
\end{itemize}
\item Rights to attend workshops and shape the \nep\ software, announced in the Slack channel.
\item Right to attend project lectures on work performed by the community, and on
relevant background material such as the spectral/hp element method, announced in the Slack channel.
\item The {\tt tex} subdirectory~\cite{xpntexwebsite} contains bibtex databases to aid report writing in subdirectory {\tt bib} and graphics suitable for producing presentations
in subdirectories {\tt pics} and {\tt png}.
\end{itemize}
The Slack channel is `\#excalibur-neptune'. (The Slack communication
software is downloadable from \url{https://slack.com/}).

(Note that access to most \nep \ reports is restricted to community members.)

\subsection{Convention on use of IETF Keywords}
The RFC2119 subset of the Internet Engineering Task Force~(IETF) keywords~\cite{rfc2119}
is used throughout the website, unless stated otherwise. Such usage implies specific
meanings for ``MUST", ``MUST NOT", ``REQUIRED", ``SHALL", ``SHALL
NOT", ``SHOULD", ``SHOULD NOT", ``RECOMMENDED",  ``MAY", and ``OPTIONAL"
\emph{when} the words are capitalised.

%
